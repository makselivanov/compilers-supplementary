\documentclass{article}

\usepackage{amssymb, amsmath}
\usepackage{alltt}
\usepackage{pslatex}
\usepackage{epigraph}
\usepackage{verbatim}
\usepackage{latexsym}
\usepackage{array}
\usepackage{comment}
\usepackage{makeidx}
\usepackage{listings}
\usepackage{indentfirst}
\usepackage{verbatim}
\usepackage{color}
\usepackage{url}
\usepackage{xspace}
\usepackage{hyperref}
\usepackage{stmaryrd}
\usepackage{amsmath, amsthm, amssymb}
\usepackage{graphicx}
\usepackage{euscript}
\usepackage{mathtools}
\usepackage{mathrsfs}
\usepackage{multirow,bigdelim}
\usepackage{subcaption}
\usepackage{placeins}

\makeatletter

\makeatother

\definecolor{shadecolor}{gray}{1.00}
\definecolor{darkgray}{gray}{0.30}

\def\transarrow{\xrightarrow}
\newcommand{\setarrow}[1]{\def\transarrow{#1}}

\def\padding{\phantom{X}}
\newcommand{\setpadding}[1]{\def\padding{#1}}

\def\subarrow{}
\newcommand{\setsubarrow}[1]{\def\subarrow{#1}}

\newcommand{\trule}[2]{\dfrac{#1}{#2}}
\newcommand{\crule}[3]{\frac{#1}{#2},\;{#3}}
\newcommand{\withenv}[2]{{#1}\vdash{#2}}
\newcommand{\trans}[3]{{#1}\transarrow{\padding{\textstyle #2}\padding}\subarrow{#3}}
\newcommand{\ctrans}[4]{{#1}\transarrow{\padding#2\padding}\subarrow{#3},\;{#4}}
\newcommand{\llang}[1]{\mbox{\lstinline[mathescape]|#1|}}
\newcommand{\pair}[2]{\inbr{{#1}\mid{#2}}}
\newcommand{\inbr}[1]{\left<{#1}\right>}
\newcommand{\highlight}[1]{\color{red}{#1}}
\newcommand{\ruleno}[1]{\eqno[\scriptsize\textsc{#1}]}
\newcommand{\rulename}[1]{\textsc{#1}}
\newcommand{\inmath}[1]{\mbox{$#1$}}
\newcommand{\lfp}[1]{fix_{#1}}
\newcommand{\gfp}[1]{Fix_{#1}}
\newcommand{\vsep}{\vspace{-2mm}}
\newcommand{\supp}[1]{\scriptsize{#1}}
\newcommand{\sembr}[1]{\llbracket{#1}\rrbracket}
\newcommand{\cd}[1]{\texttt{#1}}
\newcommand{\free}[1]{\boxed{#1}}
\newcommand{\binds}{\;\mapsto\;}
\newcommand{\dbi}[1]{\mbox{\bf{#1}}}
\newcommand{\sv}[1]{\mbox{\textbf{#1}}}
\newcommand{\bnd}[2]{{#1}\mkern-9mu\binds\mkern-9mu{#2}}
\newtheorem{lemma}{Lemma}
\newtheorem{theorem}{Theorem}
\newcommand{\meta}[1]{{\mathcal{#1}}}
\renewcommand{\emptyset}{\varnothing}
\newcommand{\dom}[1]{\mathtt{dom}\;{#1}}
\newcommand{\primi}[2]{\mathbf{#1}\;{#2}}

\definecolor{light-gray}{gray}{0.90}
\newcommand{\graybox}[1]{\colorbox{light-gray}{#1}}

\lstdefinelanguage{lama}{
keywords={read, write, skip,if,then,else,elif,fi,while,do,od,repeat,until,for,fun,local,public,return,import,length,
string,case,of,esac,when,boxed,unboxed,string,sexp,array,infix,infixl,infixr,at,before,after,true,false,eta,lazy,ignore, ref, elemRef},
sensitive=true,
basicstyle=\small,
%commentstyle=\scriptsize\rmfamily,
keywordstyle=\ttfamily\bfseries,
identifierstyle=\ttfamily,
basewidth={0.5em,0.5em},
columns=fixed,
fontadjust=true,
literate={->}{{$\to$}}3,
morecomment=[s][\ttfamily]{(*}{*)},
morecomment=[l][\ttfamily]{--}
}

\lstset{
mathescape=true,
basicstyle=\small,
identifierstyle=\ttfamily,
keywordstyle=\bfseries,
commentstyle=\scriptsize\rmfamily,
basewidth={0.5em,0.5em},
fontadjust=true,
escapechar=!,
language=lama
}

\sloppy

\newcommand{\lama}{$\lambda\kern -.1667em\lower -.5ex\hbox{$a$}\kern -.1000em\lower .2ex\hbox{$\mathcal M$}\kern -.1000em\lower -.5ex\hbox{$a$}$\xspace}

\theoremstyle{definition}

\author{Dmitry Boulytchev}

\begin{document}

\section{Arrays}

Arrays are added at the expression level by means of the following extension:

\[
\begin{array}{rcl}
  \mathscr E & += & \llang{[}\mathscr E^*\llang{]} \\
             &    & \mathscr E \llang{[$\mathscr E$]} \\
             &    & \mathscr E \;\llang{. length} \\
             &    & \llang{elemRef}\; \mathscr E\llang{[$\mathscr E$]}
\end{array}
\]

Additional well-formedness rules for the new constructs:

\ifdefined\Ref
  \renewcommand{\Ref}{\primi{Ref}{}}
\else 
  \newcommand{\Ref}{\primi{Ref}{}}
\fi
\newcommand{\Val}{\primi{Val}{}}
\newcommand{\Void}{\primi{Void}{}}
\newcommand{\Weak}{\primi{Weak}{}}

\[
\begin{array}{ccc}
  \trule{\withenv{\Val}{e_1}\,...\,\withenv{\Val}{e_k}}{\withenv{\Val}{\llang{[}e_1,...,e_k\llang{]}}} &
  \trule{\withenv{\Val}{e_1}\,...\,\withenv{\Val}{e_k}}{\withenv{\Weak}{\llang{[}e_1,...,e_k\llang{]}}} &
  \trule{\withenv{\Val}{e_1}\,...\,\withenv{\Val}{e_k}}{\withenv{\Void}{\llang{ignore [}e_1,...,e_k\llang{]}}} \\[5mm]
  \trule{\withenv{\Val}{e}\quad\withenv{\Val}{i}}{\withenv{\Val}{\llang{$e\;$[$\;i\;$]}}} &
  \trule{\withenv{\Val}{e}\quad\withenv{\Val}{i}}{\withenv{\Weak}{\llang{$e\;$[$\;i\;$]}}} &
  \trule{\withenv{\Val}{e}\quad\withenv{\Val}{i}}{\withenv{\Void}{\llang{ignore $\;e\;$[$\;i\;$]}}}\\[5mm]
  \multicolumn{3}{c}{\trule{\withenv{\Val}{e}\quad\withenv{\Val}{i}}{\withenv{\Ref}{\llang{elemRef $\;e\;$[$\;i\;$]}}}}
\end{array}
\]

\section{Operational Semantics}

An array can be represented as a pair: the length of the array and a mapping from indices to elements. If we denote
$X$ the set of elements then the set of all arrays $\mathscr A (\mathscr X)$ can be defined as follows:

\[
\mathscr A(X) = \mathbb N \times (\mathbb N \to X)
\]

For an array $\inbr{n,\, f}$ we assume $\dom{f}=[0\,..\,n-1]$. An element selection function:

\[
\begin{array}{c}
  \bullet[\bullet] : \mathscr A (X) \to \mathbb N \to X\\[2mm]
  \inbr{n,\, f}\, [i] = \left\{
                  \begin{array}{rcl}
                     f (i) &, & i < n\\
                     \bot&,&\;\mbox{otherwise}
                  \end{array}
               \right.
\end{array}
\]

We represent arrays by \emph{references}. Thus, we introduce a (linearly) ordered set of locations

\[
\mathscr L = \{l_0, l_1, \dots\}
\]

Now, the set of all values the programs operate on can be described as follows:

\[
    \mathscr V = \mathbb Z \mid \mathscr L 
\]

To access arrays, we introduce an abstraction of memory:

\[
    \mathscr M = \mathscr L \to \mathscr A\,(\mathscr V)
\]

We now add two more components to the configurations: a memory function $\mu$ and the first free memory location $l_m$, and
define the following primitive

\[
\primi{mem}{\inbr{\sigma,\,\omega,\,\mu,\,l_m}}=\mu
\]

which gives a memory function from a configuration.

The definition of state does not change, hence all existing rules are preserved (modulo adding additinal components to configurations)
The rules for the new kinds of expressions are as follows:

\setarrow{\xRightarrow}
\setsubarrow{_{\mathscr E}}
\[
\trule{\setsubarrow{_{\mathscr E^*}}
      \trans{c}{e_0,...,e_k}{\inbr{\inbr{\sigma,\,\omega,\,\mu,\,l},\,v_1,...,v_k}}}
      {\trans{c}{\llang{[}e_0,...,e_k\llang{]}}{\inbr{\inbr{\sigma,\,\omega,\,\mu[l\gets\inbr{k+1,\,i\mapsto v_i}],\,l+1},\,l}}}\ruleno{Array}
\]
\[
\trule{\setsubarrow{_{\mathscr E^*}}\trans{c}{ei}{\inbr{c^\prime,\,lv}}\quad l\in\mathscr L\quad v\in\mathbb Z}
      {\trans{c}{\llang{$e\;$[$\;i\;$]}}{\inbr{c^\prime,\,((\primi{mem}{c^\prime})(l))[i]}}}\ruleno{ArrayElem}
\]
\[
\trule{\setsubarrow{_{\mathscr E^*}}\trans{c}{ei}{\inbr{c^\prime,\,lv}}\quad l\in\mathscr L\quad v\in\mathbb Z}
      {\trans{c}{\llang{elemRef$\;e\;$[$\;i\;$]}}{\inbr{c^\prime,\,\primi{elemRef}{l\;v}}}}\ruleno{ArrayElemRef}
\]
\[
\trule{\trans{c}{e}{\inbr{c^\prime,\,l}}\quad l\in \mathscr L}
      {\trans{c}{\llang{$e\;$. length}}{\inbr{c^\prime,\,\primi{fst}{(\primi{mem}{c^\prime})(l)}}}}\ruleno{ArrayLength}
\]

We also need one additional rule for assignment:

\[
\trule{\setsubarrow{_{\mathscr E^*}}\trans{c}{lr}{\inbr{\inbr{\sigma,\,\omega,\,\mu,\,l},\,[\primi{elemRef}{a\;i}]v}}\quad a\in\mathscr L}
      {\trans{c}{\llang{$l\;$ := $\;r$}}{\inbr{\inbr{\sigma,\,\omega,\,\mu[a\gets\inbr{\primi{fst}{\mu\,(a)},\,(\primi{snd}{\mu\,(a)})[i\gets v]}],\,l},\,v}}}\ruleno{AssignArray}
\]


\section{Stack Machine}

In stack machine we add the following new instructions:

\[
\begin{array}{rcl}
  \mathscr I & += & \llang{ARRAY}\;\mathbb N \\
             &    & \llang{ELEM} \\
             &    & \llang{STA}
\end{array}
\]

We also add memory function and current location components to the configuration; as state components are preserved, all rules are
preserved as well. The new rules are:

\setsubarrow{_{\mathscr{SM}}}
\[
\trule{\withenv{P}{\trans{\inbr{s[n,..],\,s_c,\,\sigma,\,\omega,\,\mu[l\gets\inbr{n,\,i\mapsto s[n-i-1]}],\,l+1}}{p}{c}}}
      {\withenv{P}{\trans{\inbr{s,\,s_c,\,\sigma,\,\omega,\,\mu,\,l}}{[\llang{ARRAY $\;n$}]p}{c}}}\ruleno{ARRAY$_{\mathscr{SM}}$}
\]
\[
\trule{\withenv{P}{\trans{\inbr{[(\mu\,(a))[i]]s,\,s_c,\sigma,\,\omega,\,\mu,\,l}}{p}{c}}}
      {\withenv{P}{\trans{\inbr{ias,\,s_c,\,\sigma,\,\omega,\,\mu,\,l}}{[\llang{ELEM}]p}{c}}}\ruleno{ELEM$_{\mathscr{SM}}$}
\]
\[
\trule{\withenv{P}{\trans{\inbr{vs,\,s_c,\sigma,\,\omega,\,\mu[a\gets\inbr{\primi{fst}{(\mu\,(a))},\,(\primi{snd}{(\mu\,(a))})[i\gets v]}],\,l}}{p}{c}}}
      {\withenv{P}{\trans{\inbr{vias,\,s_c,\,\sigma,\,\omega,\,\mu,\,l}}{[\llang{STA}]p}{c}}}\ruleno{STA$_{\mathscr{SM}}$}
\]
\end{document}
